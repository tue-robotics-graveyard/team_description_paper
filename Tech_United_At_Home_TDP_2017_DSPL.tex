
%%%%%%%%%%%%%%%%%%%%%%% file typeinst.tex %%%%%%%%%%%%%%%%%%%%%%%%%
%
% This is the LaTeX source for the TDPTemplate using
% the LaTeX document class 'llncs.cls' Springer LNAI format
% used in the RoboCup Symposium submissions.
% http://www.springer.com/computer/lncs?SGWID=0-164-6-793341-0
%
% It may be used as a template for your own TDP - copy it
% to a new file with a new name and use it as the basis
% for your Team Description Paper
%
% NB: the document class 'llncs' has its own and detailed documentation, see
% ftp://ftp.springer.de/data/pubftp/pub/tex/latex/llncs/latex2e/llncsdoc.pdf
%
%%%%%%%%%%%%%%%%%%%%%%%%%%%%%%%%%%%%%%%%%%%%%%%%%%%%%%%%%%%%%%%%%%%

\documentclass[runningheads,a4paper]{llncs}

% document input
\usepackage[utf8]{inputenc}		% input encoding
\usepackage[english]{babel}		% input language for hyphens

% fonts
\usepackage[T1]{fontenc}		% more glyphs and a
\usepackage{lmodern}			% better looking font

% page geometry
%\usepackage{a4wide}
%\usepackage{a4}
%\usepackage[left=48mm,right=46mm]{geometry}
\usepackage[left=32mm,right=31mm]{geometry}

\usepackage{amssymb}
\usepackage{amsmath}
\setcounter{tocdepth}{3}


\usepackage{hyperref}
\usepackage{graphicx}
\usepackage{caption}
\usepackage{subcaption}
\captionsetup{compatibility=false}
\usepackage{float}

\usepackage{url}
\usepackage{listings}
\usepackage{enumitem}
\usepackage{textcomp}

% *** MORE GRAHPICS ***
\usepackage[usenames,dvipsnames]{color}

% *** BIBLIOGRAPHY PACKAGES ***
%\usepackage{natbib}

%%%%%%%%%%%%%%%%%%%%%%%%%%%%%%%%%%%%%%%%%%%%%%%%%%%%%%%%%%%%%%%%%%%
\usepackage{booktabs}           % For tables (toprule, midrule, bottomrule)
\usepackage{todonotes}			% should be defined after the color package
%%%%%%%%%%%%%%%%%%%%%%%%%%%%%%%%%%%%%%%%%%%%%%%%%%%%%%%%%%%%%%%%%%%

%%%%%%%%%%%%%%%%%%%%%%%%%%%%%%%%%%%%%%%%%%%%%%%%%%%%%%%%%%%%%%%%%%%
% *** PATHS ***
\makeatletter
\def\input@path{{Figures/}		
				}
\makeatother

\graphicspath{  {Figures/}
				}
%%%%%%%%%%%%%%%%%%%%%%%%%%%%%%%%%%%%%%%%%%%%%%%%%%%%%%%%%%%%%%%%%%%

%%%%%%%%%%%%%%%%%%%%%%%%%%%%%%%%%%%%%%%%%%%%%%%%%%%%%%%%%%%%%%%%%%%

% Acronym definitions
\usepackage[acronym]{glossaries}
\newacronym{ed}{ED}{Environment Descriptor}
\newacronym{amcl}{AMCL}{Adaptive Monte Carlo Localization}
\newacronym{gui}{GUI}{Graphical User Interface}
%\newacronym{spl}{SPL}{Standard Platform League}
\newacronym{fcfg}{FCFG}{feature context free grammar}
\newacronym{ros}{ROS}{Robot Operating System}
\newacronym{wire}{WIRE}{World Information for Robotic Environments}
\newacronym{cnn}{CNN}{Convolution Neural Networks}

% shorthand definitions
\newcommand{\eg}{\emph{e.g.}}						% Exemplum gratia
\newcommand{\goal}{\mathcal{G}}						% Goal area
\newcommand{\goallc}{\mathcal{G}_{\mathrm{lc}}}		% Subset of goal area with costs below threshold cmin
\newcommand{\goalhc}{\mathcal{G}_{\mathrm{hc}}}		% Subset of goal area with costs above threshold cmin
\newcommand{\ie}{\emph{i.e.}}						% Id est
%%%%%%%%%%%%%%%%%%%%%%%%%%%%%%%%%%%%%%%%%%%%%%%%%%%%%%%%%%%%%%%%%%%

\begin{document}

\title{Tech United Eindhoven @Home \\2017 Team Description Paper}
\author{M.F.B.~van~der~Burgh, J.J.M.~Lunenburg, R.P.W.~Appeldoorn, R.W.J.~Wijnands, T.T.G.~Clephas, M.J.J.~Baeten, L.L.A.M.~van~Beek, R.A.~Ottervanger, H.W.A.M.~van~Rooy and M.J.G.~van~de~Molengraft}
\institute{Eindhoven University of Technology,
\newline Den Dolech 2, P.O. Box 513, 5600 MB Eindhoven, The Netherlands\\
\texttt{http://www.techunited.nl, techunited@tue.nl, https://github.com/tue-robotics}}
\authorrunning{Tech United Eindhoven}

%\author{Team Leader \and Team Members }
%\institute{[Intitute name and direction here], \\
%\texttt{http://devoted-web-site.url}}
\maketitle


%%%%%%%%%%%%%%%%%%%%%%%%%%%%%%%%%%%%%%%%%%%%%%%%%%%%%%%%%%%%%%%%%%%%%%%%%%%%%%%%%%%%

\begin{abstract}
This paper provides an overview of the main developments of the Tech United Eindhoven RoboCup@Home team. Tech United uses an advanced world modeling representation system called the Environment Descriptor that allows straight forward implementation of localization, navigation, exploration, object detection \& recognition, object manipulation and robot-robot cooperation skills. Recently developments are improved object detection via deep learning methods, a generic GUI for different user levels, improved speech recognition and improved natural language interpretation. These developments are done on AMIGO and SERGIO and will now also be implemented on the Toyota HSR.
\end{abstract}

%%%%%%%%%%%%%%%%%%%%%%%%%%%%%%%%%%%%%%%%%%%%%%%%%%%%%%%%%%%%%%%%%%%%%%%%%%%%%%%%%%%%

\section{Introduction}
Tech United Eindhoven\footnote{\url{http://www.techunited.nl}} is the RoboCup student team of Eindhoven University of Technology\footnote{\url{http://www.tue.nl}} that (since 2005) successfully competes in the robot soccer Middle Size League (MSL) and later (2011) joined the ambitious @Home League. The Tech United @Home team is the vice champion of RoboCup 2016 in Leipzig and the reigning European Champion of the 2016 RoboCup European Open. The robot soccer middle-size Tech United team has an even greater track record with 3 world championship titles. See the Tech United website for more results. Tech United Eindhoven consists of (former) PhD and MSc students and staff members from different departments within the Eindhoven University of Technology.
\\\\
This Team Description Paper is part of the qualification package for RoboCup 2017 in Nagoya, Japan and describes the current status of the @Home activities of Tech United Eindhoven. The main achievement of our long-term development is our generic world model \acrshort{ed}. Recent developments are improved object detection via deep learning methods, a generic GUI for different user levels, improved speech recognition and improved natural language interpretation. 

\section{\acrfull{ed}}
The TUe \acrfull{ed} is a \acrfull{ros} based 3D geometric, object-based world representation system for robots. In itself ED is database system that structures multi-modal sensor information and represents this in an object-based world representation that can be utilized for robot localisation, navigation, manipulation and interaction functions. See Figure \ref{fig:ed} for a schematic overview of ED. %\footnote{\acrshort{ed} is an evolution of \acrfull{wire}, that was created in the FP7 RoboEarth Project. Secondly, ED is utilized within the RoboCup @home competition (also read the \href{https://github.com/tue-robotics/team_description_paper/blob/master/Tech_United_At_Home_TDP_2015.pdf}{TU/e 2015 TDP} for RoboCup). More information, software, installation manual and tutorial can be found on \url{https://github.com/tue-robotics/ed}}
%An elaborate explanation, including tutorials are available at our GitHub website \footnote{\url{http://github.com/tue-robotics}}.
ED is used on our robots AMIGO and SERGIO in the open @Home league and will be used on the Toyota HSR in the DSPL. In previous years, developments have been focussed towards making ED platform independent. As a results ED had been used on the PR2 system, Turtlebot and Dr. Robot systems (X80).
\begin{figure}[h]
    %\vspace{-0.3cm}
	\includegraphics[width = 0.9\linewidth]{Figures/ed_overview}
    %\vspace{-1em}
	\caption{schematic overview of TUe Environment Descriptor.}
	\label{fig:ed}
    %\vspace{-0.5cm}
\end{figure}
ED is one re-usable environment description that can be used for a multitude of needed functionalities. Instead of having different environment representations for localization \acrfull{amcl}, navigation (MoveBase), manipulation (MoveIt!), interaction, etc.. An improvement in this single, central world model will reflect in the performances of the separate robot capabilities. It omits updating and synchronization of multiple world models. At the moment different ED modules exist that enable robots to localize themselves, update positions of known objects based on recent sensor data, segment and store newly encountered objects and visualize all this through a web-based \acrshort{gui}, illustrated in Figure \ref{fig:gui_actions}.
\begin{figure}[h]
\centering
    %\vspace{-0.3cm}
	\includegraphics[width = 0.8\linewidth]{Figures/ed_segmentation}
    %\vspace{-0.5em}
	\caption{A view of the world model created with ED. The figure show the occupation grid as well as (unknown) objects recognized on top of the cabinet.}
	\label{fig:ed_segmentation}
    %\vspace{-0.5cm}
\end{figure}



\subsection{Localization, Navigation and Exploration}
The \acrshort{ed}-localization\footnote{\url{https://github.com/tue-robotics/ed_localization}} plugin implements \acrshort{amcl} based on a 2D render from the central world model. In order to navigate, a model of the environment is required. This model is stored in the (\acrshort{ed}). From this model, a planning representation is derived that enables using the model of the environment for navigation purposes.
\\
With use of the ed\_navigation plugin \footnote{\url{https://github.com/tue-robotics/ed_navigation}}, an occupancy grid is derived from the world model and published as a nav\_msgs/OccupancyGrid. This grid can be used by a motion planner to perform searches in the configuration space of the robot.
\\
With the use of the cb\_base\_navigation ROS package\footnote{\url{https://github.com/tue-robotics/cb_base_navigation}}. The robots are able to deal with end goal constraints. With use of a ROS service, provided by the ed\_navigation plugin, an end goal constraint can be constructed w.r.t. a specific world model entity described by ED. This enables the robot to not only navigate to poses but also to areas or entities in the scene, as illustrated by Figure \ref{fig:ed_navigation_constraints}. Somewhat modified versions of the local and global ROS planners available within move\_base are used. 
\begin{figure}[h]
	\centering
	%\vspace{-0.2cm}
	\includegraphics[width = 1\linewidth]{Figures/ed_navigation_constraints}
	%\vspace{-0.5em}
	\caption{Navigation position constraints w.r.t. other entities in the environment}
	\label{fig:ed_navigation_constraints}
	%\vspace{-0.5cm}
\end{figure}

\subsection{Object detection}
\subsubsection{Detection \& Segmentation}
ED enables integrating sensor data with use of the plugins present in the ed\_sensor\_integration package. Two different plugins do exist:
1. laser\_plugin: Enables tracking of 2D laser clusters. This plugin can be used to track dynamic obstacles such as humans.
2. kinect\_plugin: Enables world model updates with use of data from the Microsoft Kinect\texttrademark. This plugin exposes several ROS services that realize different functionalities:
\begin{enumerate}[label=(\alph*)]
\item Segment: Service that segment sensor data that is not associated with other world model entities. Segmentation areas can be specified per entity in the scene. This allows to segment object ‘on-top-of’ or ‘in’ a cabinet.
\item FitModel: Service that fits the specified model in the sensor data of the Microsoft Kinect\texttrademark. This allows updating semi-static obstacles such as tables and chairs.
\end{enumerate}


The ed\_sensor\_integration plugins enable updating and creating entities. However, new entities are classified as unknown entities.
\begin{figure}[h]
    \centering
    %\vspace{-0.3cm}
	\includegraphics[width = 1\linewidth]{Figures/ed_perception}
    %\vspace{-1em}
    \caption{ED Perception responsible for the object segmentation and calling the object recognition service. Left, the segmented objects in the robot's sensor frame are displayed; the final annotated world representation is shown at the right picture.}
	\label{fig:ed_perception}
    %\vspace{-0.5cm}
\end{figure} 

\subsection{Object grasping, moving and placing}
As for manipulating objects, the architecture is only focused on grasping. The input is the specific target entity in the world model \acrshort{ed}. The output is the grasp motion, i.e. joint positions for all joints in the kinematic chain over time. Figure \ref{fig:grasping_pipeline} shows the grasping pipeline.
\begin{figure}[h]
    \centering
    \vspace{-0.3cm}
	\includegraphics[width = 0.9\linewidth]{Figures/grasping_pipeline}
    \vspace{-1em}
	\caption{Grasping pipeline.}
	\label{fig:grasping_pipeline}
    \vspace{-0.5cm}
\end{figure}
A python executive queries the current pose of the entity from \acrshort{ed}. The resulting grasp pose goes to the grasp precompute component which makes sure that we approach the object in a proper way. MoveIt will produces joint trajectories over time with use of the current configuration, the URDF model and the final configuration. Note that MoveIt currently does not take any information from \acrshort{ed} into account. Finally, the trajectories are sent to the reference interpolator which sends the trajectories either to the controllers or the simulated robot. 

\section{Image Recognition}
In order the classify or train unknown entities, the ed\_perception plugin\footnote{\url{https://github.com/tue-robotics/ed_perception}} exposes ROS Services to classify the entities in the world model. The ed\_perception module interfaces with various image\_recognition nodes that apply state of the art image classification techniques based on \acrfull{cnn} illustrated in Figure \ref{fig:cnn}.
\begin{figure}[H]
    \centering
    %\vspace{-0.3cm}
	\includegraphics[width = 1\linewidth]{Figures/cnn}
    %\vspace{-1em}
    \caption{Illustration \acrfull{cnn} used in our object recognition nodes with use of Tensorflow.}
	\label{fig:cnn}
    %\vspace{-0.5cm}
\end{figure}

\subsection{Object recognition using Deep Learning}
Object recognition is done using Tensorflow: retraining the top-layer of a Inception V3 neural network. The top layers are retrained on a custom dataset using a soft-max top-layer that maps the image representation on a specified set of labels.
\\
In order to create a new training set for specific objects, the ed\_perception and the image\_recognition packages contains several tools for segmenting and annotating objects. Also tools for retraining neural networks are included.

\subsection{Face recognition}
Face detection and recognition is done using Openface based on Torch. Openface is an existing state-of-the-art face recognition library. We implemented a ROS node that enables the use of these advanced technologies within the ROS network.
\subsection{ROS packages}
Our image recognition ROS packages can be found at GitHub\footnote{\url{https://github.com/tue-robotics/image_recognition}} with tutorials and documentation. Recently, they have also been added to the ROS Kinetic package list and can be installed as Debian packages:
\begin{lstlisting}
ros-kinetic-image-recognition
\end{lstlisting}

%\subsection{Reasoning}
%The reasoning layer of the AMIGO ROS based software consists of a set of finite state machines (robot\_smach\_states1) that build upon the robot’s skill layer (robot\_skills2). These state machines are useful if you want the robot to execute some complex plan, where all possible states and state transitions can be described explicitly. 
The implementation is done with use of the open-source SMACH3 package. SMACH is a task-level architecture for rapidly creating complex robot behaviours. At its core, SMACH is a ROS-independent Python library to build hierarchical state machines. SMACH is a new library that takes advantage of very old concepts in order to quickly create robust robot behaviour with maintainable and modular code.

%\newpage
\section{Human-Robot Interface}
\subsection*{Overview}

In order to interact with the robot aside of speech, a web-based Graphical User Interface (GUI) has been designed. An HTML5 website\footnote{\texttt{https://github.com/tue-robotics/tue\_mobile\_ui}} is hosted on the robotic platform that offers a GUI to multiple users on different platforms with use of a Robot API\footnote{\texttt{https://github.com/tue-robotics/robot-api}} implemented in Javascript. Figure~\ref{fig:webgui_architecture} shows an overview of how the user can interact with the robot via this interface. ToDo: what kind of interaction? --> can you give some examples? Maybe say what the action server does?

\begin{figure}[ht]
        \includegraphics[width = \linewidth]{webgui_architecture}
        \caption{Overview webGUI architecture. The robot's functionalities are exposed with use of the Robot API that is implemented in javascript. The Webserver that is hosting the GUI connects this Robot API to a graphical user interface that is offered to multiple clients on different platforms.}
        \label{fig:webgui_architecture}
\end{figure}




\section{Robot Descriptions}
\subsection{Robot Hardware Descriptions}
At the moment of writing of this team description paper, we are not in the possession of the Toyota HSR yet. Therefore we are not able to inform you about any possible additional computing devices in this paper. 

\subsection{Robot Software Description}
An overview of the software used by the Tech United Eindhoven @Home robots can be found in Table~\ref{tab:softwarespec}.
All our software is developed open-source at GitHub\footnote{\url{https://github.com/tue-robotics}}.
\\\newline
Currently, we have some \textit{image\_recognition} packages released into the current ROS Kinetic distribution and can be installed with use of \textit{apt}.

\begin{table}[H]
    \begin{center}
    \caption{Software overview of the robots.}
    \label{tab:softwarespec}
    %\vspace{-0.1cm}
    \renewcommand{\arraystretch}{1.0}
    \setlength{\tabcolsep}{5pt}
        \begin{tabular}{p{0.3\textwidth} p{0.7\textwidth}}
            \toprule
            Operating system & Ubuntu 16.04 LTS Server\\

            Middleware & ROS Kinetic~\cite{Quigley2009}\\

            Low-level control software & Orocos Real-Time Toolkit~\cite{Bruyninckx2001}\newline
            \url{https://github.com/tue-robotics/rtt_control_components}
            \\

            Simulation & Custom kinematics + sensor simulator \newline
            \url{https://github.com/tue-robotics/fast_simulator}
            \\

            World model & \acrfull{ed}, custom \newline
            \url{https://github.com/tue-robotics/ed}\\

            Localization & Monte Carlo~\cite{Fox2003} using \gls{ed}, custom \newline \url{https://github.com/tue-robotics/ed\_localization}\\

            SLAM & Gmapping package \newline \url{http://wiki.ros.org/gmapping}\\

            Navigation & CB Base navigation
            \newline
            \url{https://github.com/tue-robotics/cb_base_navigation}
            \newline
            Global: custom A* planner\newline Local: modified ROS DWA~\cite{Fox1997}\\

            Arm navigation & Custom implementation using MoveIt and Orocos KDL\newline
            \url{https://github.com/tue-robotics/tue_manipulation}
            \\

            Object recognition & Tensorflow ROS \newline
			\url{https://github.com/tue-robotics/image\_recognition/tree/master/tensorflow\_ros}\\

            People detection & Custom implementation using contour matching \newline
            \url{https://github.com/tue-robotics/ed_perception}
            \\
            Face detection \& recognition & Openface ROS \newline \url{https://github.com/tue-robotics/image\_recognition/tree/master/openface\_ros} \\

            Speech recognition & Julius Speech Recognition \newline
            \url{https://github.com/julius-speech/julius}\\
            Speech synthesis & Toyota\texttrademark \hspace{0em} Text-to-Speech\\
            Task executors & SMACH \newline
            \url{https://github.com/tue-robotics/tue_robocup}\\
            \bottomrule
        \end{tabular}
    \end{center}
\end{table}

This our current software implementation on our robots AMIGO and SERGIO, which participate in the open league. Because of the pending delivery of the Toyota HSR and related documentation, we are not able to provide the specific software implementation. As described in our selection qualification paper, our goal is to use the same software as possible on all our robots, including the Toyota HSR.


\subsection{Re-usability of the system for other research groups}
Tech United takes great pride in creating and maintaining open-source software and hardware to accelerate innovation. Tech United initiated the \href{http://roboticopenplatform.org/}{Robotic Open Platform website}, to share hardware designs. All packages are equipped with documentation and tutorials. Tech United and its scientific staff have the capacity to co-develop (+10 people), maintain and assist with questions. 

\subsection{Community Outreach and Media}
The Tech United team carries out many promotional activities to promote technology and innovation with children. These activities are carried out by separate teams of student assistants. Tech United often visits primary and secondary schools, public events, trade fairs and have regular TV performances. In 2015 and 2016 together, 100+ demos were given and an estimated 50k were reached through live interaction.
Tech United has also got a very active (\href{www.techunited.nl}{website}, and interacts on many social media mediums: \href{https://www.facebook.com/techunited}{Facebook}, \href{https://www.youtube.com/user/TechUnited}{YouTube}, \href{https://twitter.com/TechUnited}{Twitter} and \href{https://www.flickr.com/photos/techunited/}{Flickr}. Our robotics videos are often shared on the IEEE video Friday website.

\bibliographystyle{unsrt}
\bibliography{refs}

\end{document}
