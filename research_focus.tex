The TUe Control Systems Technology (CST) group has an internationally recognized reputation. CST targets areas in precision machines, robotics, biomedical, agriculture and automotive engineering. New controller and observer synthesis methods are being developed and applied to create innovative intelligent systems. The recent scientific visitation of the group resulted in the highest achievable score (excellent).
\\
Specific fields of robotics research are: world modelling, context awareness, motion planning, whole-body motion control, active perception, sensor fusion, distributed control (swarms).
\\
CST was coordinator of the world wide acclaimed \href{http://roboearth.ethz.ch/}{FP7 project RoboEarth}, proposing a system design for the internet of robots. The Tech United team directly benefits from currently running robotics research projects: \href{http://www.r5-cop.eu/en/}{R5COP} (resilient and reconfigurable robot SW/HW), \href{https://www.eureyecase.eu/}{EurEyeCase} (robot assisted eye-surgery), \href{http://cordis.europa.eu/project/rcn/206247_en.html}{ROPOD} (warehouse robots) and \href{http://cordis.europa.eu/project/rcn/201768_en.html}{AUTOPILOT} (autonomous driving).
\\
Our recent research efforts have focused on i) fitting furniture objects to update our object-oriented world model, thereby improving localization, navigation and object segmentation ii) developing a WebGUI to provide the user with a platform-independent way to interact with the robot, iii) natural language interpretation to ease the specification of written commands for the robot and to make speech recognition more robust and iv) improved image recognition using neural network training via \href{https://github.com/tue-robotics/image_recognition/tree/master/tensorflow_ros}{tensorflow\_ros}.
\\
Our recent scientific contributions can be found on the \href{http://www.techunited.nl/en/artikel}{Tech United website}
