The \acrshort{ed}-localization\footnote{\url{https://github.com/tue-robotics/ed_localization}} plugin implements \acrshort{amcl} based on a 2D render from the central world model. In order to navigate, a model of the environment is required. This model is stored in the (\acrshort{ed}). From this model, a planning representation is derived that enables using the model of the environment for navigation purposes.
\\
With use of the ed\_navigation plugin \footnote{\url{https://github.com/tue-robotics/ed_navigation}}, an occupancy grid is derived from the world model and published as a nav\_msgs/OccupancyGrid. This grid can be used by a motion planner to perform searches in the configuration space of the robot.
\\
With the use of the cb\_base\_navigation ROS package\footnote{\url{https://github.com/tue-robotics/cb_base_navigation}}. The robots are able to deal with end goal constraints. With use of a ROS service, provided by the ed\_navigation plugin, an end goal constraint can be constructed w.r.t. a specific world model entity described by ED. This enables the robot to not only navigate to poses but also to areas or entities in the scene, as illustrated by Figure \ref{fig:ed_navigation_constraints}. Somewhat modified versions of the local and global ROS planners available within move\_base are used. 
\begin{figure}[h]
	\centering
	%\vspace{-0.2cm}
	\includegraphics[width = 1\linewidth]{Figures/ed_navigation_constraints}
	%\vspace{-0.5em}
	\caption{Navigation position constraints w.r.t. other entities in the environment}
	\label{fig:ed_navigation_constraints}
	%\vspace{-0.5cm}
\end{figure}