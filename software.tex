An overview of the software used by the Tech United Eindhoven @Home robots can be found in Table~\ref{tab:softwarespec}.
All our software is developed open-source at GitHub\footnote{\texttt{https://github.com/tue-robotics}}

\begin{table}[H]
    \begin{center}
    \caption{Software overview of the robots.}
    \label{tab:softwarespec}
    \vspace{-0.1cm}
    \renewcommand{\arraystretch}{1.0}
    \setlength{\tabcolsep}{5pt}
        \begin{tabular}{p{0.3\textwidth} p{0.7\textwidth}}
        	\toprule
            Operating system & Ubuntu 14.04 LTS Server\\

            Middleware & ROS Indigo~\cite{Quigley2009}\\

            Low-level software & Orocos Real-Time Toolkit~\cite{Bruyninckx2001}\\

            World model & \acrfull{ed}, custom \newline \url{https://github.com/tue-robotics/ed}\\

            Localization & Monte Carlo~\cite{Fox2003} using \gls{ed}, custom \newline \url{https://github.com/tue-robotics/ed_localization}\\

            SLAM & Gmapping: \texttt{http://wiki.ros.org/gmapping}\\

            Navigation & Global: custom A* planner\newline Local: modified ROS DWA~\cite{Fox1997}\\

            Arm navigation & Custom implementation using MoveIt and Orocos KDL\\

            Object recognition & Tensorflow ROS \newline \url{https://github.com/tue-robotics/image_recognition/tree/master/tensorflow_ros} \\

            People detection & Custom implementation using contour matching \\
            Face detection \& recognition & Openface ROS \newline \url{https://github.com/tue-robotics/image_recognition/tree/master/openface_ros} \\
            Speech recognition & Dragonfly + Windows Speech Recognition \newline \texttt{http://code.google.com/p/dragonfly/}\\
            Speech synthesis & Philips Text-to-Speech\\
            Task executors & SMACH\newline\texttt{http://wiki.ros.org/smach}\\
            \bottomrule
        \end{tabular}
    \end{center}
\end{table}
