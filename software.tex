An overview of the software used by the Tech United Eindhoven @Home robots can be found in Table~\ref{tab:softwarespec}.
All our software is developed open-source at \href{https://github.com/tue-robotics}{GitHub}.
%%\vspace{-0.9cm}
\begin{table}[h]
    \begin{center}
    \caption{Software overview of the robots.}
    \label{tab:softwarespec}
    %\vspace{-0.1cm}
    \renewcommand{\arraystretch}{1.0}
    \setlength{\tabcolsep}{5pt}
        \begin{tabular}{p{0.3\textwidth} p{0.7\textwidth}}
        	\toprule
            Operating system & Ubuntu 14.04 LTS Server\\

            Middleware & ROS Indigo \\

            Low-level software & Orocos Real-Time Toolkit\\

            World model & \href{https://github.com/tue-robotics/ed}{\acrfull{ed}}, custom \\

            Localization & \href{https://github.com/tue-robotics/ed_localization}{Monte Carlo using \gls{ed}}, custom \\

            SLAM & Gmapping: \texttt{http://wiki.ros.org/gmapping}\\

            Navigation & Global: custom A* planner\newline Local: modified ROS DWA \\

            Arm navigation & Custom implementation using MoveIt and Orocos KDL\\

            Object recognition & \href{https://github.com/tue-robotics/image_recognition/tree/master/tensorflow_ros}{Tensorflow ROS} \\

            People detection & Custom implementation using contour matching \\
            Face detection \& recognition & \href{https://github.com/tue-robotics/image_recognition/tree/master/openface_ros}{Openface ROS} \\
            Speech recognition & \href{http://code.google.com/p/dragonfly/}{Dragonfly} + Windows Speech Recognition\\
            Speech synthesis & Philips Text-to-Speech\\
            Task executors & SMACH \texttt{http://wiki.ros.org/smach}\\
            \bottomrule
        \end{tabular}
    \end{center}
\end{table}
%%\vspace{-1cm}
