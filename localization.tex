The \acrshort{ed}-localization plugin\footnote{\url{https://github.com/tue-robotics/ed localization}} implements Monte Carlo localization based on the environment description in ED, laserscan readings (sensor\_msgs/Laserscan) and a valid transformation between the /odom and the /base\_link frame of the robot (tf). It does not differ much from the ROS AMCL package\footnote{\url{http://wiki.ros.org/amcl}}; where the AMCL package uses a grid map as representation, the ED-localization plugin uses a 2D render from its world model. 
Slam\_gmapping\footnote{\url{http://wiki.ros.org/gmapping}} - ROS simultaneous localization and mapping (SLAM) component - was configured for usage on PR2 and Toad robots. It requires only laser scanner (messages of type sensor\_msgs/LaserScan) and odometry (TF transform between robot base and odometry reference frame). When there is a static environment as in the use case it can be used to obtain 2D map and then robot can localize against this map by the \acrshort{amcl} package. The \acrshort{amcl} also requires only laser scanner and odometry and provides reasonably robust localization. Full description of both components is given on their respective \acrshort{ros} wiki pages.